\section{Modifying the Multi-Cycle Design}
A pipelined processor requires clean separation between the different stages.
When a clock cycle begins, each stage must get some input, process it, then output the result.
The next stage gets its next cycle input from the previous stage output.

This is implemented by adding registers in between components in hardware.
Each stage reads its input register at the beginning of the cycle and writes its result to an output register which is the input register for the next stage.
Thus the clock frequency of the entire processor is dependent upon the time it takes for the slowest stage to read input, process it, then write output.

\subsection{Pipeline Stages}
Figure~\ref{fig:architecture} illustrates the main components in each pipeline stage.
The forwarding unit and the hazard detection unit is left out to simplify the figure,
and they are described in greater detail in seperate sections.

\begin{figure}[h]
    \centering
    \includegraphics[width=\textwidth]{img/architecture}
    \caption{
      Simplified overview of the architecture.
    }
    \label{fig:architecture}
\end{figure}

\subsection{Information-passing Between Stages}
With pipelining it is important to have registers to temporarily store the next value between each stage.
To achive this behaviour, registers were added between the stages:
IF/ID, ID/EX, EX/MEM and MEM/WB.
Each of the stage seperators contains registers which is used for control signals and busses.

However signals from memory needs to pass through the stage seperators without any delay.
The instruction memory and the data memory outputs are pass throug signals.

\subsection{Resources Shared by Stages}
\begin{table}[h]
    \begin{tabular}{l|lllll}
    ~         & IF & ID & EX & MEM & WB \\ \hline
    PC        & RW & ~  & ~  & W   & ~  \\
    Registers & ~  & R  & ~  & ~   & W  \\
    \end{tabular}
\end{table}
