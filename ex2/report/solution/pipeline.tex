\section{Pipelined Design}
A pipelined processor requires synchronous separation between the different stages.
When a clock cycle begins,
each stage must get some input, process it, then output the result by the end of the cycle.
The next stage gets its next cycle input from the previous stage output.

This is implemented by adding registers between the executing units of the stages.
Each stage reads its input register at the beginning of the cycle and writes its result to an output register which is the input register for the next stage.
Thus the clock frequency of the entire processor is dependent upon the time it takes for the slowest stage to read input, process it, then write output.

\subsection{Pipeline Stages}
Figure~\ref{fig:architecture} illustrates which components are used by each pipeline stage.
The forwarding unit and the hazard detection unit is left out to simplify the figure.
These modules are described in sections \ref{sec:forwarding} and \ref{sec:hazard-detection}.

\begin{figure}[h]
    \centering
    \includegraphics[width=\textwidth]{img/architecture}
    \caption{
      Simplified overview of the architecture.
    }
    \label{fig:architecture}
\end{figure}

\noindent
During development, the group used two design sketches for reference.
These are attached to the report in appendix \ref{app:architecture}.

\subsubsection{Resources Shared by Stages}
One thing to note in figure \ref{fig:architecture} is that the PC module and the registers module take inputs from multiple stages.
The PC module will select the input from MEM if a control signal is asserted,
or else it defaults to the input from the IF stage.
For the registers module,
the input from the ID stage is asking to read registers, while the input from the WB stage is writing to a register.
The registers module is designed to allow these actions concurrently.
This way structural hazards are avoided.

\subsection{Information-passing Between Stages}
The registers simplified as yellow blocks in figure \ref{fig:architecture} are implemented as separate entities of the same VHDL sources.

The outputs from the instruction and data memories are the slowest part of the pipeline.
To avoid having to wait too long between cycles,
these outputs bypass registers and will arrive directly from the output of the memory to the next stage.

\subsubsection{VHDL Register Modules}
The values that pass through the registers are of different sizes,
from 32-bit instructions and operands to single bit signals.
Registers for these values are implemented in \gls{vhdl} as generic components that can be re-used.

\begin{description}
    \item[\texttt{bit\_register}]
        \hfill\\
        Holds a single bit value.
    \item[\texttt{generic\_register}]
        \hfill\\
        Has a generic \texttt{WIDTH} property that is specified when an entity is created.
        Holds a bit vector of the specified width.
    \item[\texttt{EX\_register}, \texttt{MEM\_register}, \texttt{WB\_register}]
        \hfill\\
        These special registers hold the record control signal bundles that the control module outputs.
        These registers can be seen in figure \ref{fig:suggested_architecture} labeled EX, MEM and WB.
\end{description}

