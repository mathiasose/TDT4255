\section{Modifying the Multi-Cycle Design}
A pipelined processor requires clean separation between the different stages.
When a clock cycle begins, each stage must get some input, process it, then output the result.
The next stage gets its next cycle input from the previous stage output.

This is implemented by adding registers in between components in hardware.
Each stage reads its input register at the beginning of the cycle and writes its result to an output register which is the input register for the next stage.
Thus the clock frequency of the entire processor is dependent upon the time it takes for the slowest stage to read input, process it, then write output.

\subsection{Information-passing Between Stages}
With pipelining it is important to have registers to temporarily store the next value,
between each stage.
To achive this behaviour, four registers were added between stages:
IF/ID, ID/EX, EX/MEM and MEM/WB.

However signals from memory needs to pass through the stage seperators without using a registers.
The instruction memory and the data memory outputs are pass throug signals.




Instruction Fetch => Instruction Decode
32 bit instruction
Program counter value

Instruction Decode => Execute
Program counter value
Control signals: WB, M, EX
Register read values A and B
Extended immediate value
Rt and Rd values

Execute => Memory Access
Control signals: WB, M
Modified PC value
ALU result and zero flag
Register read value B
Rt or Rd value after muxing by RegDst

Memory Access => Write to Registers
WB control signals
Data memory read value
Data memory address
Rt or Rd value after muxing by RegDst

\subsection{Resources Shared by Stages}
\begin{table}[h]
    \begin{tabular}{l|lllll}
    ~         & IF & ID & EX & MEM & WB \\ \hline
    PC        & RW & ~  & ~  & W   & ~  \\
    Registers & ~  & R  & ~  & ~   & W  \\
    \end{tabular}
\end{table}
