\section{Pipeline Hazards}
The following sections descriptibes how the processor detects and corrects various hazards.

\subsection{Structural Hazards}
Structural hazards are avoided entirely in the implemented design.
The separation of instruction memory from data memory means that the IF stage may read the instruction memory concurrently with the WB stage reading or writing the data memory.

\subsection{Data Hazards}
\label{sec:data-hazards}
\subsubsection{Forwarding}
When a result is calculated in the execute stage,
it will be several cycles until it is ready to be read from a register by a dependent instruction.
One solution to this problem is to stall the pipeline until the result is ready in the register,
but this is not the most efficient solution.

A better approach is to implement a forwarding scheme,
making results in the MEM or WB stages available in earlier stages.

A forwarding unit forwards data from MEM or WB stage, to the EX and the ID stage when needed.
Figure \ref{fig:fwd_mux} and listing \ref{lst:select_source} illustrate the forwarding unit.
This generic component is instantiated twice in the ID stage and twice in the EX stage.
A register value and its associated register address comes from a previous stage.
If the selecter unit detects that the selected register has an updated value in the MEM or WB stage,
it will reroute this value to replace the value from the earlier stage.

Figure \ref{fig:forwarders} shows the placement of the forwarders in the pipeline.

\begin{figure}[h]
    \centering
    \includegraphics[width=.6\textwidth]{img/fwd_mux}
    \caption{Forwarder multiplexer unit. Inputs on the right come from a later stage.}
    \label{fig:fwd_mux}
\end{figure}

\noindent{
    \begin{minipage}{\linewidth}
        \begin{lstlisting}[language=VHDL, label=lst:select_source, caption=Source select logic]
if MEM_reg_write = '1' and selected_register = MEM_reg_dst then
    selected_source <= MEM;
elsif WB_reg_write = '1' and selected_register = WB_reg_dst then
    selected_source <= WB;
else
    selected_source <= DEFAULT;
end if;
        \end{lstlisting}
    \end{minipage}
}

\begin{figure}[h]
    \centering
    \includegraphics[width=\textwidth]{img/forwarders}
    \caption{Simplified overview of the parts of the pipeline involving forwarding. The green arrows represent inputs from later stages, and are the same for all forwarders.}
    \label{fig:forwarders}
\end{figure}

\subsubsection{Stalling}
If there is a \textit{load word} instruction with an immediately following dependent instruction,
forwarding can not solve the hazard entirely, and a stall is required.

To enable this a hazard detection unit is implemented in the processor.
The unit detects data dependency between a \texttt{lw} instruction in the EX stage and a dependent instruction is in the ID stage.
The expression for this is shown in listing \ref{lst:hazard-detection-unit}.
If this is the case is will hold the program counter value and the instruction in the IF to ID registers for one more cycle.
The instruction passing from ID to EX will continue execution,
but is replaced in the register afterwards with a no-op to avoid executing it twice.
Once the \texttt{lw} instruction has passed the EX stage the stall condition is no longer valid,
and the pipeline continues as normal.

Figure \ref{fig:hazard-detection} shows the hazard detection unit inputs and what its output signal controls.

\begin{figure}[h]
    \centering
    \includegraphics[width=\textwidth]{img/hazard-detection}
    \caption{
      Overview of the hazard detection unit.
    }
    \label{fig:hazard-detection}
\end{figure}

\noindent{
    \begin{minipage}{\linewidth}
        \begin{lstlisting}[language=VHDL, label=lst:hazard-detection-unit, caption=Expression used to set stall control signal]
stall <= '1' when
  mem_read = '1' and (ex_rt = id_rs or ex_rt = id_rt) else '0';
        \end{lstlisting}
    \end{minipage}
}

\subsection{Control Hazards}
The implementation doesn't do any branch prediction, it just assumes the branch is not taken.
When a branch is taken and changes the execution order the pipeline before the EX stage has to be flushed,
replacing instructions with \texttt{no\_op} instructions.

In the implementation there is a \texttt{flush} signal which is indicates when the pipline has to be flushed.
Listing \ref{lst:control-hazard-detection} shows the condition for asserting the \texttt{flush} signal.

\noindent{
    \begin{minipage}{\linewidth}
        \begin{lstlisting}[language=VHDL, label=lst:control-hazard-detection, caption=Control hazard detection]
if jump = '1' or (branch = '1' and alu_zero = '1') then
    flush_pipeline <= '1';
        \end{lstlisting}
    \end{minipage}
}

The \texttt{flush} signal is connected to the reset on the registers that are to be flushed,
and the result of resetting these registers is a \texttt{no\_op}.
