\section{Five-stage MIPS Pipeline}
The \gls{mips} instruction set is designed to enable pipelining.
The pipeline that is described in this report is a typical \gls{mips} pipeline,
with the following five stages:

\begin{description}
    \item[Instruction Fetch (IF)]
        \hfill\\
        In this stage the instruction memory receives an address and outputs the corresponding instruction.
    \item[Instruction Decode (ID)]
        \hfill\\
        In this stage the instruction that was read in the IF stage is decoded.
        Control signal values are determined here that will be used in the following pipeline stages.

        It is also in this stage that registers are read,
        and the read values are passed to the next stage.
    \item[Execute Instruction (EX)]
        \hfill\\
        In this stage the \gls{alu} receives its input values and operation and outputs a result.
    \item[Memory Access (MEM)]
        \hfill\\
        In this stage a value can be either read from or written to data memory.

        If the instruction is a jump or a branch that is taken,
        this is the stage where the new program counter will be set.
    \item[Write to Registers (WB)]
        \hfill\\
        In the final stage of the pipeline, a new value may be written to its destination register.
\end{description}

