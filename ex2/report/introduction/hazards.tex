\section{Pipeline Hazards}
When implementing pipelining in a processor,
there are complications that might arise that must be handled to ensure correct results.
Instructions often depend on preceding instructions,
and if two instructions with some dependency between each other are in the pipeline at the same time,
there is a \textit{hazard} which must be handled in a some way.
\textit{Patterson \& Hennessy} lists the following pipeline hazards
\cite[Chapter 4.5]{bib:patt-henn}:

\begin{description}
\item[Structural Hazards] \hfill \\
    Instructions that access the same hardware component cannot execute in parallel.
    E.g. with a shared instruction and data memory,
    instructions cannot be fetched at the same time as data is being written.
\item[Data Hazards] \hfill \\
    Instructions that use some value (i.e. registers) depend on the preceding instructions that modify the same value.
    Executing an instruction without first receiving the new value for the dependency will yield incorrect results.
\item[Control Hazards] \hfill \\
    Also called \textbf{branch hazards}.
    When a branch instruction is executed,
    the next instruction to be executed may be one of two possibilities.
    The pipeline must be able to react to a branch and flush instructions that are not to be executed from the pipeline before the results of those instructions are committed.
\end{description}
