\section{Pipeline Hazards}
When implementing pipelining in a processor,
there are complications that might arise that must be handled to ensure correct results.
\textit{Patterson \& Hennessy} lists the following hazards
\cite[Chapter 4.5]{bib:patt-henn}:

\begin{description}
\item[Structural Hazards] \hfill \\
    Instructions that access the same hardware component cannot execute in parallel.
    E.g. with a shared instruction and data memory,
    instructions cannot be fetched at the same time as data is being written.
\item[Data Hazards] \hfill \\
    Instructions that modify some stateful object (e.g. registers) depend on the preceding instructions that also modify the same state object.
    Executing an instruction before a dependency has finished will yield erroneuous results.
\item[Control Hazards] \hfill \\
    Also called \textbf{branch hazards}.
    When a branch instruction is executed,
    the next instruction to be executed may be one of two possibilities.
    The pipeline must be able to react to a branch and flush instructions that are not to be executed from the pipeline.
\end{description}
