\section{Performance}
The piplined MIPS processor achives a higher max frequency compared to the processor implemented in the first exercise.
The processor executes one instruction each clock cycle, with a few exceptions:

\begin{itemize}
\item A \texttt{lw} instruction with a data dependent instruction afterwards, will cause the system to stall for one cycle.
\item If the \texttt{beq} expression holds true and the PC jumps, the pipline has to be flushed and a total of 3 cycles is lost.
\end{itemize}

\subsection{Futher Performance Improvements}
Most of the execution time is spent inside loops.
As a result loop optimizations often results in the best performance gains.
Branch prediction to reduce pipline flushes, may have an huge impact on performance.


It it also possible to have more than 1 instruction executed each cycle.
With multiple functional units it is possible to execute multiple instructions.
This requires extra hardware and increases complexity.
