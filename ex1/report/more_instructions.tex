\section{Adding more instructions}
\label{sec:more_instructions}
After having sucessfully tested the minimum required instruction set on the FPGA,
we took another look at the full MIPS instruction set.

\subsection{Arithmetic and logical I-type instructions}
With an already functioning implementation containing various ALU operations and I-type insructions,
it was very simple to also implement the I-type versions of the existing arithmetic-logical operations, \texttt{ADDI}, \texttt{ANDI}, \texttt{ORI} and \texttt{SLTI} by setting the correct signals in the control module.
With \texttt{NOR} and \texttt{XOR} existing as operators in VHDL, these could be implemented as ALU operations just as simply as the existing ones,
which enabled the instructions \texttt{NOR}, \texttt{XOR}, \texttt{XORI}

\subsection{Shift instructions}
In order to implement the shift instructions \texttt{SLL}, \texttt{SRL},
the ALU needed access to the part of the instruction that contains the number of positions to shift.
This signal was added and can be seen in figure \ref{fig:architecture}.
With that connection it was again a very simple task to add the shift functionality to the ALU, since VHDL provides functions for this.
