\section{Arithmetic Logic Unit}
The \gls{alu} takes two 32 bit operands.
Opcodes is recieved from the control module and the \gls{alu} performs operation based on the opcode.

\gls{alu} has a status signal attached to the \gls{pc} module called \texttt{alu\_zero}.
\texttt{alu\_zero} is used by the BEQ instruction and is set high when the SUB operation returns 0.

After implementing the required \gls{alu} operations, it required litle effort to add additional functionality.
The \gls{alu} were extended with a few extra operations: NOR, XOR, SLL, SRL, SRA.
Table \ref{table:instructions} shows all the implemented \gls{alu} functions.

\begin{table}
\centering
\begin{tabular}{ |l|l| }
  \hline
  ALU operation & Description             \\ \hline
  ADD           & Addition                \\
  SUB           & Subtraction             \\
  SLT           & Set if less than        \\
  AND           & AND                     \\
  NOR           & NOR                     \\
  XOR           & Exclusive OR            \\
  SLL           & Shift left logical      \\
  SRL           & Shift right logical     \\
  SRA           & Shift right arithmetic  \\
  \hline
\end{tabular}
\caption{Implemented ALU operations}
\label{table:instructions}
\end{table}
