\section{Arithmetic Logic Unit}
The implemented \gls{alu} can perform arithmetic, logical and shifting operations.
In each execute cycle the control module decodes the instruction to an appropriate opcode and the ALU receives this opcode as an input.
Table \ref{table:alu_ops} shows the operations implemented in the ALU.

The arithmetic and logical operations take two 32 bit operand inputs.
The first is always a register value,
the second may be either from a register or an immediate value from the instruction that has been transformed to 32-bit format.
These operations are implemented by delegating to VHDL built in operations.

Shift operations use the \texttt{shift\_amount} part of the instruction as another input.
Shifting is implemented by delegating to the standard shift functions in VHDL.
These instrutions were added later than the others,
and this is described in section \ref{sec:more_instructions}.

The \gls{alu} module outputs a signal to the \gls{pc} module called \texttt{alu\_zero} whenever the current operation yields a 0 result.

\begin{table}[h]
    \centering
    \begin{tabular}{ |l|l| }
        \hline
        ALU operation & Description             \\ \hline
        ADD           & Addition                \\
        SUB           & Subtraction             \\
        SLT           & Set if less than        \\
        AND           & AND                     \\
        NOR           & NOR                     \\
        XOR           & Exclusive OR            \\
        SLL           & Shift left logical      \\
        SRL           & Shift right logical     \\
        SRA           & Shift right arithmetic  \\
        \hline
    \end{tabular}
    \caption{Implemented ALU operations}
    \label{table:alu_ops}
\end{table}

