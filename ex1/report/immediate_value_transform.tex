\section{Immediate Value Transform Submodule}
In the suggested architecture there was indicated a module for sign extending a 16-bit vector to 32 bits.
This 16-bit vector comes from the immediate value part of I-type instructions.
At first we implemented this without a submodule,
since it only required a one-line function call in VHDL to extend the number,
as shown in listing \ref{lst:extend}.

\noindent{
    \begin{minipage}{\linewidth}
        \begin{lstlisting}[language=VHDL, label=lst:extend, caption=Extending a vector in VHDL]
extended <= operand_t(resize(unsigned(imm_val), operand_t'length));
        \end{lstlisting}
    \end{minipage}
}

As we made progress on the exercise,
we realised that the \texttt{LUI} instruction required a feature that could extend a 16-bit vector to 32 bits by left-shifting,
placed in the same location in the datapath as the sign-extending feature.
The natural solution to this was to create a new submodule which could take a 16 bit input and a 1 bit control signal, then output the appropriate extended vector.
Figure \ref{fig:imm_val_tf} shows a sketch of this module.

\begin{figure}[p]
    \centering
    \includegraphics[width=0.6\textwidth]{img/imm_val_tf}
    \caption{Sketch of immediate value transform submodule}
    \label{fig:imm_val_tf}
\end{figure}

