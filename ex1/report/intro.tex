\chapter{Introduction}
This report presents a solution to exercise 1 of the TDT4255 course at NTNU.
The objective of the exercise is to implement a multi cycle processor capable of executing a subset of the \gls{mips} instruction set, in \gls{vhdl}.
After the design simulation is verified with testbenches in \textit{XILINX ISE} it is to be synthesized and tested on \textit{XILINX Spartan-6 FPGA}.

\section{MIPS}
The instruction set for this exercise is \gls{mips} instruction set with 32 bit wide instructions.
MIPS is a RISC instruction set,
and there are three different instruction types: R(egister)-type, I(mmediate)-type and J(ump)-type.
The MIPS processor has a limited amount of fast registers near the ALU,
and uses load and store instructions to move data to and from a bigger but slower data memory.
Instructions are fetched from a memory separate from the data memory.

\section{Approach}
The top level architecture of the processor is divided into separate modules.
These modules are created by following test-driven development methods and iterative changes as the system comes together as a whole.
Once these submodules are finished they are connected and the system is tested as a whole.
An overview of the architecture can be seen in figure \ref{fig:architecture}.

