\section{Control Submodule}
The control submodule combines internal state (fetch, execute, stall) and the input from the current instruction to decide various control signals which the other submodule use as inputs.

During the fetch stage the control module does nothing except prepare to go to the next state at the next clock cycle.
It is during the execute stage that most of the output happens.
In this stage the state machine is a Mealy machine,
with the opcode and funct bits from the input instruction decide the outputs.
The outputs that are set non-default are shown in figure \ref{fig:state_machine}.

The \texttt{LW} and \texttt{SW} instructions require I/O to the data memory module,
which is a comparaively slow operation.
Therefore the stall state is required as a continuation of the execute state for these instructions.
The \texttt{pc\_write} signal controls the program counter submodule and ensures that the next instruction is not fetched before the current instruction has had enough time to execute.

\begin{figure}[h]
    \centering
    \includegraphics[width=\textwidth]{img/state_machine.png}
    \caption{State machine of control submodule.\\Binary signals default to 0, \texttt{ALU\_OP} defaults to \texttt{NO\_OP} and \texttt{imm\_val\_tf} defaults to \texttt{SIGN\_EXTEND}.}
    \label{fig:state_machine}
\end{figure}

